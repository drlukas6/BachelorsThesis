\section{Značaj podataka u dubokom učenju}
Prvi i najdulji praktični korak treninga predstavlja priprema podataka. 
Sve ovisi o zadatku koji mreža mora riješiti, ali, generalno je pravilo da je više podataka bolje.
Konačna kvaliteta rješenja osim o arhitekturi mreže koju dizajniramo, ovisi o kvaliteti podataka kojom ju usmjeravamo.
Priprema podataka vrši se u 3 glavna koraka (\cite{generalDatasets}):
\begin{enumerate}
\item Prikupljanje
\item Klasifikacija
\item Označavanje
\end{enumerate} 
\subsubsection{Prikupljanje podataka}
Prikupljanje podataka mora biti sustavan i smislen proces jer može otežati i olakšati daljnje korake. 
Najpreporučeniji način za prikupljanje je dugoročno i postepeno spremanje podataka jer rezultira velikim brojem objektivnih i kvalitetih podataka.
Ja sam se ipak odlučio na metodu računalnog generiranja vlastitog seta podataka. 
Razlog tome je raznolikost elemenata koje mreža mora moći detektirati i fleksibilnost koju dobivam jednomo kada ustanovim sve potrebe.
\subsubsection{Klasifikacija i označavanje podataka}
Generirani podaci na određeni način moraju biti prikazani mreži. 
Iako u mrežu slika ulazi kao vektor dimenzija \texttt{(visina x širina x kanali)} mreži su potrebni i podaci za uspoređivanje rezultata i računanje uspješnosti.
U ovom radu koristio sa \texttt{.csv} datoteku za dohvaćanje i kao opisnik slika. 
Postupak automatskog generiranja slika uvelike je olakšao klasifikaciju i označavanje jer je cijeli postupak ostvaren kao "cjevovod".
Pri izlasku, slika bi bila prikazana kao na slici ~\ref{fig:pipelineExitExample}.
Datoteka bi upisano imala ime slike, simbol na slici, širinu, visinu i točan položaj elementa na slici. 
Prednost ovog pristupa je i u tom što slika nije zadana absolutnom putanjom, što znači, da sam slike mogao kreirati na vlastitom računalu, prenjeti ih na udaljeni server za treniranje i bez komplikacija koristiti iste. \\
Veličina opisnika je također bila zanemariva. 
Nakon raspodjele \texttt{80:20} za trening i validaciju na 15 000 slika, veličine su bile \texttt{440kB} i \texttt{110kB} dok je direktorij sa slikama bio veličine \texttt{6,7GB}.

\begin{figure}[h!]
	\centering
	\includegraphics[width=1.0\linewidth]{image_csv}
	 \caption{Slika i pripadajuća referenca u .csv datoteci}
 	 \label{fig:pipelineExitExample}
\end{figure}

\section{Generiranje slika}
\subsection{Generalizacija postupka}
Za relativan uspjeh treniranja mreže za detekciju i klasifikaciju 14 tekstovnih elemenata (0-9, +, -, *, :) potrebno je minimalno 10 000 slika. Ne samo zbog broja elemenata već i zbog složenosti i raznolikosti između njih. 
Postupak koji sam razvio primjenjuje sve taktike (\cite{chollet2017deep}) potrebne za stvaranje raznovrsnog i kvalitetnog seta podataka.
Zbog transformacija, opisanih u daljnjim djelovima poglavlja, gotovo je nemoguće da iako se isti font stavlja na pozadinu, nastane isti oblik.
Na slici ~\ref{fig:imageGenerationPipeline} prikazana je topologija cjevovoda koja kreira slike.
Cijeli cjevovod, implementiran je unutar programskog paketa \emph{ImageGenerator}, kojeg sam napisao u svrhu apstraktiranja cijelog postupka.
\begin{figure}[h!]
	\centering
	\includegraphics[width=1.0\linewidth]{image_generation_pipeline}
	 \caption{Prikaz visoke razine cjevovoda za generiranje slika}
 	 \label{fig:imageGenerationPipeline}
\end{figure}

\subsection{Prikupljanje fontova}
Ispisivanje velikog broja simbola sa razlikom između varijacija istog monoton je i neisplativ posao, posebice zbog dostupnosti svih potrebnih resursa na internetu.
U prilog je također išlo to što su dostupni fontovi, koji primjenjuju rukopisni stil, najčešće zbilja napisani rukom i vektorizirani, pa, generiranje i transformiranje neće negativno utjecati na kvalitetu.
Osim rukopisnih fontova, skinuo sam mali broj fontova koji su stilski između čistog rukopisnog i tipkanog.
Fontovi su bili prikupljeni sa sljedećih izvora, a na slici ~\ref{fig:fontDiffs} vidljivi su primjeri istih:
\begin{itemize}
\item \url{https://www.dafont.com}
\item \url{https://www.1001fonts.com}
\item \url{https://www.1001freefonts.com}
\end{itemize}
\begin{figure}[h!]
	\centering
	\includegraphics[width=0.85\linewidth]{font_diffs}
	 \caption{Varijacije unutar simbola uzrokovane fontovima}
 	 \label{fig:fontDiffs}
\end{figure}

\subsection{Generiranje simbola}
Nakon prikupljanja i sortiranja fontova, generiranje samih simbola bio je jednostavan posao.
Važno je bilo očuvati transparentnost pozadine iza simbola jer u trenutku kada se postavi na pozadinu po izboru, ona mora biti vidljiva.
\begin{algorithm}
\caption{Generiraj sve simbole}
\begin{algorithmic}[1]
	\Function{generirajTextSliku}{$font, simbol$}
		\State $velicina \gets font.velicina(simbol)$
		\State $slika \gets Image('RGBA', velicina, (255, 255, 255, 0))$
		\State $slika.text = simbol$
		\State \Return $slika$
	\EndFunction
	\Function{generirajSveSimbole}{\null}
		\For{\texttt{simbol in simboli}}
			\For{\texttt{font in fontovi}}
				\State $direktorijSimbola \leftarrow spoji(putDoSimbola, simbol)$
				\If{$direktorijSimbola\ ne\ postoji$}
					\State \Call{kreirajDirektorij}{$direktorijSimbola$}
				\EndIf
				\State $generirana slika \gets$ \Call{generirajTextSliku}{$font, simbol$}
				\State \Call{spremiSliku}{$generiranaSlika$}
			\EndFor
		\EndFor
	\EndFunction
\end{algorithmic}
\end{algorithm}

\subsection{Transformacije}
Prije postavljanja simbola na nasumično odabranu sliku, svaki simbol prošao je kroz tri točke transformiranja:
\begin{enumerate}
\item Skaliranje
\item Rotacija
\item Afina transformacija
\end{enumerate}
Cilj transformacija je maksimalno unjeti raznolikost unutar dataseta u slučaju premalog ili presličnog broja slika.
Klasa \emph{ImageGenerator} za to se brine na sličan način kao programski paket \emph{Keras.preprocessing.image.ImageDataGenerator} (\cite{Keras.io}).
Transformaciju nad slikama vršio sam pomoću programskog paketa \emph{OpenCV} (\cite{OpenCV}) jer apstraktira potrebne matematičke operacije na razumljiv, lako koristiv i prilagodljiv način.
Tijekom faze transformiranja i postavljanja slike na pozadinu, one su u obliku matrice, definirane pomoću programskog paketa \emph{Numpy}.
\subsubsection{Skaliranje}
Skaliranje pomoću \emph{OpenCV} paketa može se vršiti ili ručno, specifirajući točnu veličinu, ili dajući faktor skaliranja.
\emph{OpenCV} također automatski primjenjuje \emph{interpolaciju} kako bi se kvalitete maksimalno sačuvala.
Skaliranje se vrši na način da se matrica slike pomnoži sa matricom skaliranja, zadanom na sljedeći način:
$$
M
=
\begin{bmatrix}
	s_{x} & 0 \\
	0 & s_{y}
\end{bmatrix},
$$
gdje je $s_{x}$ faktor skaliranja u $x$ dimenziji, a $s_{y}$ faktor skaliranja u $y$ dimenziji.
Rezultat skaliranja vidljiv je na slici ~\ref{fig:scaling}
\lstset{numbers=left}
\lstinputlisting[language=python]{CodeSamples/Scaling.py}
\begin{figure}[h!]
	\centering
	\includegraphics[width=0.4\linewidth]{Scaling}
	 \caption{Rezultat primjene skaliranja sa $s_{x} = s_{y} = 1.25$}
 	 \label{fig:scaling}
\end{figure}
\subsubsection{Rotacija}
Rotacija slike, za kut $\theta$ ostvaruje se množenjem sa matricom rotacije:
$$
M
=
\begin{bmatrix}
	\cos(\theta) && -\sin(\theta) \\
	\sin(\theta) && \cos(\theta)
\end{bmatrix}
$$
Iako sam ja rotiranje vršio iz središnje točke, \emph{OpenCV} nudi podršku za eksplicitno zadavanje točke oko koje će se rotacija vršiti.
Rezultat rotiranja vidljiv je na slici ~\ref{fig:Rotating}
\lstset{numbers=left}
\lstinputlisting[language=python]{CodeSamples/Rotation.py}
\begin{figure}[h!]
	\centering
	\includegraphics[width=0.4\linewidth]{Rotation}
	 \caption{Rezultat primjene rotacije sa $\theta = 25$}
 	 \label{fig:Rotating}
\end{figure}
\subsubsection{Afine transfromacije}
Afine transformacije koristim za prividno transformiranje simbola "u prostoru", bez velikog rizika od prevelike distorzije slike jer sve paralelne linije, nakon transformacije ostaju paralelne.
\emph{OpenCV} Afinu transformaciju vrši tako da tri odabrane točke na slici pomakne za određeni koeficijent.
Kao i ostale transformacije, matematički nastaje množenjem matrice slike s matricom afine transformacije oblika.
$$
\begin{bmatrix}
	1 && \tan(\beta) \\
	\tan(\alpha) && 1
\end{bmatrix}
$$
gdje su $\alpha$ i $\beta$ razlike u kutevima prema pripadajućim koordinatnim osima.
Rezultat primjene afine transformacije na simbolu vidljiv je na slici ~\ref{fig:Affine}
\lstset{numbers=left}
\lstinputlisting[language=python]{CodeSamples/Affine.py}
\begin{figure}[h!]
	\centering
	\includegraphics[width=0.4\linewidth]{Affine}
	 \caption{Rezultat primjene afine transformacije s pripadajućim referentnim točkama}
 	 \label{fig:Affine}
\end{figure}

\subsection{Kreiranje cjelovitih slika}
Kreiranje cjelovitih slika svodilo se na postavljanje generiranih i transformiranih simbola na pozadinu po izboru.
Pozadina također igra veliku ulogu u prepoznavanju jer mreža pregledava cijelu sliku. 
Za svoje potrebe odlučio sam za pozadinu koristiti slike matematičkih bilježnica uz pretpostavku da bi se iste koristile naviše kada bi se ova mreža koristila u stvarnom svijetu.
Izlazi iz mreže vidljivi su na slikama ~\ref{fig:ImageGeneratorOutputs} i ~\ref{fig:ImageGeneratorTermOutputs}.
Slike su spremljene u direktorij \texttt{Images}, a \texttt{.csv} opisnik u direktorij \texttt{Data} odakle će se dalje referencirati za kreiranje \texttt{.record} datoteke za daljnje korištenje \emph{Tensorflow-u}.
\begin{figure}
	\subfloat[Izlazna slika iz \emph{ImageGenerator-a}] {%
		\includegraphics[width=0.4\linewidth]{image_generator_output} %
		\label{fig:ImageGeneratorOutputs}
	}

	\subfloat[Ispis za praćenje statusa generiranja slika] {%
		\includegraphics[width=1.0\linewidth]{image_generator_term_output} %
		\label{fig:ImageGeneratorTermOutputs}
	}
\end{figure}

\section{TFRecords}
Zašto je bolje za mrežu da podatke čita iz \texttt{.record} datoteke nego odvojeno slike i pripadajuće opise?
Zamislimo sljedeći scenarij.
Treniranje se vrši na računalu sa \texttt{HDD} diskom, slike i oznake su u različitim direktorijima.
Svako čitanje sljedeće slike i oznake rezultira potencijalnim pomicanjem glave diska.
Cilj je da sve potrebne datoteke budu što bolje poravnate u memoriji.
Tu se pokazuje najveći značaj \emph{TFRecords} datoteke. 
Jedna binarna datoteka koja sadrži sve informacije za mrežu, jedinstveno poravnata u memoriji (\cite{TFRecords}). \\
U pozadini, \emph{TFRecords} je format koji koristi \emph{Protocol buffer} tehnologiju.
\emph{Protocol buffer} ili kraće \emph{Protobuf} je knjižnica za efikasnu serijalizaciju strukturiranih podataka (\cite{tensorflow.org}).
Konkretno, koristimo \emph{Protobuf} poruke oblika \texttt{"string" : value} za predstavljanje objekata mreži.
U mom slučaju, slike su zapisane na sljedeći način:
\begin{itemize}
\item height = \texttt{int64}
\item width = \texttt{int64}
\item filename = \texttt{bytes}
\item sourceid = \texttt{bytes}
\item encoded = \texttt{bytes}
\item format = \texttt{bytes}
\item xmins = \texttt{float\_list}
\item xmaxs = \texttt{float\_list}
\item ymins = \texttt{float\_list}
\item ymaxs = \texttt{float\_list}
\item classes\_text = \texttt{bytes\_list}
\item classes = \texttt{int64\_list}
\
\end{itemize}
Svi navedeni podaci zapisuju se pod \texttt{feature} ključ. \\
Kako u našem slučaju vršimo detekciju i klasifikaciju objekata, bitno je da na neki način i klasama damo jedinstveni identifikator.
Naime, u \emph{TFRecords} datoteku pod ključ \emph{classes} koji sadrži podatke o tom koji su svi objekti na slici ne pišemo doslovno ime objekta (npr. Automobil, kuća, ...).
Pišemo brojčanu vrijednost istog objekta koja ga predstavlja. 
Isti način je precizniji i sažetiji. \\
Konkretno ime dnevnika koji sadrži mapiranja iz objekta u njegovu brojčanu vrijednost naziva se \emph{Label map} i osim za stvaranje \emph{TFRecords} datoteke, koristi ga i sama mreža i mi kad iz mreže čitamo što je ista prepoznala.
Za kreiranje datoteke pratio sam korake opisane na službenoj \emph{Tensorflow} stranici.
