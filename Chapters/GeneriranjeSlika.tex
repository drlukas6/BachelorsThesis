\section{Značaj podataka u dubokom učenju}
Prvi i najdulji praktični korak treninga predstavlja priprema podataka. 
Sve ovisi o zadatku koji mreža mora riješiti, ali, generalno je pravilo da je više podataka bolje.
Konačna kvaliteta rješenja osim o arhitekturi mreže koju dizajniramo, ovisi o kvaliteti podataka kojom ju usmjeravamo.
Priprema podataka vrši se u 3 glavna koraka (\cite{generalDatasets}):
\begin{enumerate}
\item Prikupljanje
\item Klasifikacija
\item Označavanje
\end{enumerate} 
\subsubsection{Prikupljanje podataka}
Prikupljanje podataka mora biti sustavan i smislen proces jer može otežati i olakšati daljnje korake. 
Najpreporučeniji način za prikupljanje je dugoročno i postepeno spremanje podataka jer rezultira velikim brojem objektivnih i kvalitetih podataka.
Ja sam se ipak odlučio na metodu računalnog generiranja vlastitog seta podataka. 
Razlog tome je raznolikost elemenata koje mreža mora moći detektirati i fleksibilnost koju dobivam jednomo kada ustanovim sve potrebe.
\subsubsection{Klasifikacija i označavanje podataka}
Generirani podaci na određeni način moraju biti prikazani mreži. 
Iako u mrežu slika ulazi kao vektor dimenzija \texttt{(visina x širina x kanali)} mreži su potrebni i podaci za uspoređivanje rezultata i računanje uspješnosti.
U ovom radu koristio sa \texttt{.csv} datoteku za dohvaćanje i opisnik slika. 
Postupak automatskog generiranja slika uvelike je olakšao klasifikaciju i označavanje jer je cijeli postupak ostvaren kao "cjevovod".
Pri izlasku, slika bi bila prikazana kao na slici ~\ref{fig:pipelineExitExample}.
Datoteka bi upisano imala ime slike, simbol na slici, širinu, visinu i točan položaj elementa na slici. 
Prednost ovog pristupa je i u tom što slika nije zadana absolutnom putanjom, što znači, da sam slike mogao kreirati na vlastitom računalu, prenjeti ih na udaljeni server za treniranje i bez komplikacija koristiti iste. \\
Veličina opisnika je također bila zanemariva. 
Nakon raspodjele \texttt{80:20} za trening i validaciju na 15 000 slika, veličine su bile \texttt{440kB} i \texttt{110kB} dok je direktorij sa slikama bio veličine \texttt{6,7GB}.

\begin{figure}[h!]
	\centering
	\includegraphics[width=1.0\linewidth]{image_csv}
	 \caption{Slika i pripadajuća referenca u .csv datoteci}
 	 \label{fig:pipelineExitExample}
\end{figure}

\section{Generiranje slika}
\subsection{Generalizacija postupka}
Za relativan uspjeh treniranja mreže za detekciju i klasifikaciju 14 tekstovnih elemenata (0-9, +, -, *, :) potrebno je minimalno 10 000 slika. Ne samo zbog broja elemenata već i zbog složenosti i raznolikosti između njih. 
Postupak koji sam razvio primjenjuje sve taktike (\cite{chollet2017deep}) potrebne za stvaranje raznovrsnog i kvalitetnog seta podataka.
Zbog transformacija, opisanih u daljnjim djelovima poglavlja, gotovo je nemoguće da iako se isti font stavlja na pozadinu, nastane isti oblik.
Na slici ~\ref{fig:imageGenerationPipeline} prikazana je topologija cjevovoda koja kreira slike.

\begin{figure}[h!]
	\centering
	\includegraphics[width=1.0\linewidth]{image_generation_pipeline}
	 \caption{Prikaz visoke razine cjevovoda za generiranje slika}
 	 \label{fig:imageGenerationPipeline}
\end{figure}
\subsection{Prikupljanje fontova}

\subsection{Generiranje simbola}

\subsection{Transformacije}
\subsubsection{Skaliranje}
\subsubsection{Rotacija}
\subsubsection{Afine transfromacije}

\subsection{Kreiranje cjelovitih slika}
