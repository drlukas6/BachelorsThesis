Ovaj rad obuhvatio je cjelokupni cjevovod jednog ciklusa pripreme duboke neuronske mreže za specifičan zadatak.
Proces stvaranja slika omogućio je gotovo neograničen skup podataka za neuronsku mrežu.
Isti skup podataka korišten je za za prenamjenjivanje \emph{SSD} neuronske mreže za detekciju na zadatak prepoznavanja rukom napisanih simbola.
U detalje je opisana arhitektura spomenute mreže i metode koje koristi za računanje kvalitete izlaza.

Inicijalno, mreža je bila naučena za prepoznavanje objekata unutar $6000$ kategorija.
Dalje je korištena metoda prenamjenjivanja mreže na druge kategorije koristeći postojeće težine.
Spomenuta pretpostavka uvelike je olakšala i ubrzala cijeli postupak jer mnogi primitivi koji postoje unutar tih $6000$ kategorija, pojavljuju se i u simbolima.

Nakon procesa učenja u $15 000$ koraka tj. $80$ epoha, dosegnuta je zadovoljavajuća točka preciznosti i pouzdanosti na večini simbola.
Jedini simbol, koji nije uspješno naučen je simbol množenja što je i jedan od budućih izazova.
Ostvariti zadovoljavajuću preciznost na svim simbolima i vršiti segmentaciju po retcima.
Također, preporučuje se i korištenje drugih arhitektura (npr. \emph{Faster R-CNN}) te usporedba rezultata.
 
